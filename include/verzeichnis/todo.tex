%---------------------------------------------------------------------------------------------------
% Todo-Verzeichnis
%---------------------------------------------------------------------------------------------------

%	Float-Paket, um eigene Umgebungen zu definieren (für Todos benötigt)
\usepackage{float}
%	Float-Umgebung für Todos definieren
\newfloat{ToDo}{hbtp}{todoslist}		%	Verwendung:	\newfloat{<klasse>}{<position>}{<erw>}[<ebene>]
%	neuer Zähler für Nummerierung der Todos
\newcounter{todonr}
%	Todo-Markierung bauen
\newcommand{\todo}[1]{
	\addtocounter{todonr}{1}
	\FloatBarrier
	\begin{ToDo}
		%
		\vspace{-8pt}
		%\boxed{	%	Rahmen ziehen, um Layout zu prüfen und zu korrigieren
			%	zweispaltig mit minipage, damit Todo-Icon und Todo-Text vertikal mittig gegeneinander ausgerichtet werden
			\begin{minipage}{0.09\textwidth}
				\includegraphics[width=.8cm]{./img/todo.jpg}
			\end{minipage}
			\begin{minipage}{0.9\textwidth}
				\textbf{Todo\,\arabic{todonr}:} #1
			\end{minipage}
			%	Eintrag in das Todo-Verzeichnis hinzufügen, welches mit \tableoftodos erzeugt wird
			\addcontentsline{todoslist}{ToDo}{#1}
		%}
	\end{ToDo}
	\vspace{-6pt}
	\FloatBarrier
}
%	TOC-Befehl für Todo-Verzeichnis
\newcommand{\tableoftodos}{
	\addtocounter{page}{-2}
	\listof{ToDo}{TODOs}
	\newpage
}
% Layout für Todo-Verzeichnis anpassen
\makeatletter
\renewcommand*{\listof}[2]{%
  \@ifundefined{ext@#1}{\float@error{#1}}{%
    \expandafter\let\csname l@#1\endcsname \l@figure  % <- use layout of figure
    \float@listhead{#2}%	Headline
    \begingroup%
      \@starttoc{\@nameuse{ext@#1}}%
    \endgroup}}%
\makeatother
