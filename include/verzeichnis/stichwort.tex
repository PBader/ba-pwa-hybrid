%---------------------------------------------------------------------------------------------------
% Stichwortverzeichnis
%---------------------------------------------------------------------------------------------------

\usepackage{makeidx}
\renewcommand{\indexname}{Stichwortverzeichnis}

%	Erstellung des Stichwortverzeichnisses
\makeindex

%	fügt dem Index ein Wort hinzu
\newcommand{\ind}[1]{#1\index{#1}%
%	%	Ausgabe des Begriffs am Seitenrand
%	\marginpar{%
%		%\ifthenelse{\isodd{\value{page}}}{}{\hspace{\evensidemargin}}%		macht vertikalen Abstand, ungefähr eine Leerzeile --> damit verschiebt sich der Text am Seitenrand um eine Zeile tiefer als das eigentliche Wort (vermutlich wegen der links-/rechts-wechselnden Seiten bei geraden und ungeraden Seitenzahlen)
%		\parbox[t]{1.5\oddsidemargin}{%
%			\hspace{0pt}\em%
%			%	wenn Seitenzahl ungerade, dann linksbündig, ansonsten rechtsbündig
%			%\ifthenelse{\isodd{\value{page}}}{\raggedright{#2}}{\raggedleft{#2}}%
%			\raggedright{#1}%
%		}%
%	}%
}
%	Formatierungen, um zu vermeiden, dass der Index auf einer neuen Seite ausgegeben wird
\makeatletter
	\renewenvironment{theindex}{
%		%\section{\indexname}\label{sec-index}%		Überschrift vermeiden, eigene/s Section/Kapitel soll verwendet werden können
%		\@mkboth{\MakeUppercase\indexname}%
%		{\MakeUppercase\indexname}%
%		\thispagestyle{plain}\parindent\z@
%		\parskip\z@ \@plus .3\p@\relax
%		\columnseprule \z@
%		\columnsep 35\p@
%		\let\item\@idxitem

%		\twocolumn%[\section*{\indexname}]%		%	2-spaltiges Layout setzen	(besser multicol-package nutzen, deshalb auskommentiert)
		\@mkboth{\MakeUppercase\indexname}%		
		{\MakeUppercase\indexname}%
		\thispagestyle{plain}\parindent\z@
		\parskip\z@ \@plus .3\p@\relax
		\columnseprule \z@
		\columnsep 35\p@
		\let\item\@idxitem
	}{}%{\onecolumn}		%	1-spaltiges Layout wieder herstellen	(stattdessen einfach nur {}, weil \twocolumn nicht verwendet wird)
\makeatother
