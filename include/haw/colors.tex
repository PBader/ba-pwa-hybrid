%---------------------------------------------------------------------------------------------------
% offizielle HAW-Farben verwenden
%---------------------------------------------------------------------------------------------------


% Festlegen der Farben gemäß Corporate-Design-Manual der HAW
% siehe http://www.bui.haw-hamburg.de/fileadmin/redaktion/Intranet/Logos_und_CD/Design_Manual.pdf
% hier gefunden: http://www.bui.haw-hamburg.de/215.html
\definecolor{farbeHAWHamburg}{RGB}{10, 31, 99}
\definecolor{farbeHAWHamburgStudierendenzentrum}{RGB}{156, 188, 218}


% Einfärben von Section-Überschriften
% Formatierung wurde aus der phdthesis.sty übernommen und überschrieben (lediglich Verwendung der Farbe hinzugefügt, restliche Formatierung entspricht phdthesis.sty)
\titleformat{\section}[hang]{\color{farbeHAWHamburg}\sffamily\bfseries}{\Large\thesection}{12pt}{\Large}[{\color{farbeHAWHamburgStudierendenzentrum}\titlerule[0.5pt]}]

% Einfärben von Subsection-Überschriften
% (auch ein bisschen Neuformatierung, phdthesis formatiert nur \section, hier erhält auch \subsection den Sans Serif Font)
\titleformat*{\subsection}{\large\bfseries\sffamily\color{farbeHAWHamburg}}


% Einfärben von Chapter-Überschriften
% Chapter-Formatierung wird von quotchap-Package übernommen, dieses wird von phdthesis.sty eingebunden
%\renewcommand*\sectfont{\color{farbeHAWHamburg}}
% Einfärben von Chapter-Numerierungen
% wird vom quotchap-Package verwendet (bei fehlender Angabe wird Default verwendet, also grau)
\definecolor{chaptergrey}{RGB}{156, 188, 218}

