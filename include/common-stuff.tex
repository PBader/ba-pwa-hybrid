%---------------------------------------------------------------------------------------------------
% Diverse Pakete einbinden
%---------------------------------------------------------------------------------------------------

%	Titelbild
%\usepackage[cc]{./include/titlepic}


%	American Mathematical Society (Pakete für mathematischen Formelsatz --> Klammersetzungen, Rechenopreatoren, Konstanten, Fonts etc.)
\usepackage{amsmath}
\usepackage{amsfonts}
\usepackage{amssymb}


%	Unicode UTF8
\usepackage{ucs}
\usepackage[utf8x]{inputenc}
%	deutsches Sprachpaket
\usepackage{ngerman}


%	optischer Randausgleich
\usepackage[activate]{pdfcprot}


%	Farben
\usepackage{color}
%	Bilder
\usepackage{graphicx}
%  FloatBarrier
\usepackage{placeins}


%	Farbige Tabellen (Zeilen, Spalten und Zellen)
\usepackage{colortbl}
\usepackage[table]{xcolor}


%	Bild-, Tabellen und sonstige Unterschriften mit vom Standard abweichendem vertikalen Abstand zum Element darüber und anderer Schriftgröße
\usepackage[skip=8pt,font=small]{caption}


%	Einbinden externer PDF-Dateien
\usepackage[final]{pdfpages}
%	Verwendung (Achtung, ohne .pdf-Endung!)
%	\includepdf{verzeichnis/pdf-datei}

%	mehr-spaltige Layouts (\twocolumn und \onecolumn klappen zwar, erzeugen aber auch immer ein Pagebreak, multicol macht das nicht)
\usepackage{multicol}

%	absolute Positionierung von Elementen auf der Seite
\usepackage[absolute,overlay]{textpos}
%	Verwendung
%	\begin{textblock}{⟨hsize⟩}[⟨ho⟩,⟨vo⟩](⟨hpos⟩,⟨vpos⟩) text... \end{textblock}


%	Symbole, Sonderzeichen etc.
%	siehe auch http://carroll.aset.psu.edu/pub/CTAN/info/symbols/comprehensive/symbols-a4.pdf
\usepackage{pifont}
%	Verwendung: \ding{<NUM>}
%	wobei <NUM> der numerische Index für das jeweilige Symbol ist


%---------------------------------------------------------------------------------------------------
% einige eigene Befehle
%---------------------------------------------------------------------------------------------------

%	Tüddelchen
\newcommand{\anf}[1]{\glqq{}#1\grqq}
\newcommand{\anfstriche}[1]{\anf{#1}}	%	Alias
%	alternativ einfach "`das hier"' verwenden
%	(ein Command hat lediglich den Vorteil, dass man die Anführungsstriche bei Bedarf leichter austauschen kann, z.B. für andere Sprachen --> dürfte jedoch eher selten vorkommen...)

%	Ausgabe von Steuereingaben des Benutzers, z.B. für Manuals
%	Taste
\newcommand{\key}[1]{\huge\keystroke{#1}\normalsize}
%	Maus
\newcommand{\mouse}[1]{\huge\ComputerMouse\normalsize{} #1}

%	Verantwortlichkeiten als Randbemerkung
%\newcommand{\verantwortung}[1]{\marginpar{\tiny{Verantwortlich: #1}}}

%	Ausgabe von LaTeX-Commands im Dokument
\newcommand{\commandNoParam}[1]{\textup{\textmd{\texttt{\textbackslash{}#1\{\}}}}}
\newcommand{\commandOneParam}[2]{\textup{\textmd{\texttt{\textbackslash{}#1\{#2\}}}}}
\newcommand{\commandTwoParams}[3]{\textup{\textmd{\texttt{\textbackslash{}#1\{#2\}\{#3\}}}}}
\newcommand{\commandThreeParams}[4]{\textup{\textmd{\texttt{\textbackslash{}#1\{#2\}\{#3\}\{#4\}}}}}
\newcommand{\commandFourParams}[5]{\textup{\textmd{\texttt{\textbackslash{}#1\{#2\}\{#3\}\{#4\}\{#5\}}}}}
\newcommand{\commandFiveParams}[6]{\textup{\textmd{\texttt{\textbackslash{}#1\{#2\}\{#3\}\{#4\}\{#5\}\{#6\}}}}}

%	http://tex.stackexchange.com/questions/2441/how-to-add-a-forced-line-break-inside-a-table-cell#19678
%	Achtung: Texte werden horizontal zentriert, nur vertikale kann angegeben werden
%\newcommand{\td}[2][c]{%
%	\begin{tabular}[#1]{@{}c@{}}#2\end{tabular}
%}
%	-->	erstmal auskommentiert, weil im Moment doch noch keine Verwendung dafür besteht, vorsorglich aber drin gelassen...


%	Häkchen ("Ja") und Kreuzchen ("Nein") für Feature-Matrizen
\definecolor{farbeJa}{rgb}{0.1, 0.45, 0.3}
\definecolor{farbeNein}{rgb}{0.5, 0.1, 0.1}
\definecolor{farbeUnbekannt}{rgb}{0.8, 0.6, 0.1}
\newcommand{\ja}{\color{farbeJa}{\ding{52}}}
\newcommand{\nein}{\color{farbeNein}{\ding{56}}}
\newcommand{\unb}{\color{farbeUnbekannt}{\ding{108}}}

